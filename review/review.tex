\documentclass[letterpaper]{article}
\usepackage{natbib,alifexi}
\usepackage{longtable}

\title{Review: Immediate action is the best strategy when facing uncertain climate change}
\author{Jan van den Schilden$^{1}$, Nil Fernandez Lojo$^{1}$, Borel Kamdem$^{1}$ \and Siddheswar Mukherjee$^2$ \\
\mbox{}\\
$^1$Université Libre de Bruxelles, Belgium\\
$^2$Vrije Universiteit Brussel, Belgium }

\begin{document}
\maketitle

 \section{Review}
This is a professionally written report.
The introduction is very well written with the context and relevance of the work clearly stated.
The text is written in a reversed pyramidal fashion.
It start very broad, introducing the social dilemma of climate change,
and that it is often modeled by game theory as collective-risk game.
They continue in explaining the shortcomings of these collective-risk games 
that traditionally model this climate change dilemma.
Next, they propose a solution to remedy these shortcomings.
Finally, they end with a paragraph that summarises what we can expect to find in their paper.
The introduction reads very fluently 
and questions are asked at the right moments.
Enough background is given to understand the relevance of their presented work and the concepts behind it. 
In short, Clearly written introduction.

The Methods are very well described and it is clear how to reproduce their work.
Not only do they write the methods in clean pseudo-code,
they also provide a link to their Github repository,
making it possible to exactly replicate their work.
The methods are explained both in a conceptual and mathematical manner, 
making it easy to follow.
Finally, their example at the end was very helpful 
and helped to clear out any remaining ambiguity.

The authors reproduced the results of their selected paper and also added a relevant extension to their work.
As minor remark, we want to note that is would have been better 
to move the explanation about risk curves to the methods part as it is rather a method than a result.
Also a more elaborate definition of "wait and see".
Very nice figures. 
All questions that were asked in the introduction were answered and most of them were explained clearly.

The first paragraph of the conclusion nicely recapitulates the main problem that is discussed in this paper.
Further paragraphs explain in more details the answers to the questions asked in the introduction.
It is enjoyable how the authors portrayed the results as real world concepts and not just in terms of game theory.
We think they also could have added a paragraph in the discussion that explained the limits of the model and how this could be extended.
But in general,
the conclusion answer and explain the questions that were presented in the introductions.

There is a clean overall style.
Introduction was written in reversed pyramidal fashion.
Methods were explained in different ways:
as concepts, in mathematical form, in pseudocode, and with an example.
Results were provided both in text and as self contained figure. 
Conclusion nicely recapitulates the general problems and gives some further explanation.

\section{Positive Points}
\begin{enumerate}
    \item Clear introduction, and very interesting and enjoyable that the methods were explained on different levels: as concepts, in mathematical form, in pseudocode, and by aid of an example.
    \item Clear explanation of the reasons for each choice in the design of the game environment.
    \item They make an additional contribution by exploring the case in which there is a return on investment factor.
\end{enumerate}

\section{Negative Points}
\begin{enumerate}
    \item Statistical significance of the results: you could have done multiple simulations and add the empirical standard deviations of your results.
    \item Method Extension: The ROI percentage is drawn from a uniform distribution; however the authors do not mention the reason for the choice of such a distribution.
    \item Minor spelling and grammatical mistakes were encountered in the introduction and conclusion sections.
\end{enumerate}

\section{Questions}
\begin{enumerate}
    \item Do you believe that having 2 players and 4 rounds is representative of the climate change situation. Maybe adding more rounds and more players could change the results and be more significant for climate change.
    \item For the extension, wouldn't it be more relevant to add some risk for investment? Otherwise players could invest all their leftover money at each round since they know they will get back at the next round with some additional profit.
    \item Why do you think that the agent converge toward a situation that prioritizes the risk before investment? Could you change the parameters of the model in a way that would result in the agent converging toward self interest?  Why does the extreme case where each players invests all of their money in each round not occur? Would that not give more return?
\end{enumerate}

\end{document}

